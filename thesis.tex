% Opcje klasy 'iithesis' opisane sa w komentarzach w pliku klasy. Za ich pomoca
% ustawia sie przede wszystkim jezyk i rodzaj (lic/inz/mgr) pracy, oraz czy na
% drugiej stronie pracy ma byc skladany wzor oswiadczenia o autorskim wykonaniu.
\documentclass[english,shortabstract,mgr]{iithesis}

\usepackage[utf8]{inputenc}

%%%%% DANE DO STRONY TYTUŁOWEJ
% Niezaleznie od jezyka pracy wybranego w opcjach klasy, tytul i streszczenie
% pracy nalezy podac zarowno w jezyku polskim, jak i angielskim.
% Pamietaj o madrym (zgodnym z logicznym rozbiorem zdania oraz estetyka) recznym
% zlamaniu wierszy w temacie pracy, zwlaszcza tego w jezyku pracy. Uzyj do tego
% polecenia \fmlinebreak.
\polishtitle    {Tłumaczenie współczesnego języka programowania\fmlinebreak do Maszyny Minsky'ego.}
\englishtitle   {Modern programming language translation to theoretical model of Minsky Machine (Counter Machine).}
\polishabstract {TODO: polishabstract}
\englishabstract{
Counter Machine (a machine with finite state set, two counters and input/output tape) is~able to express any
computations done by~modern programming languages. This is~well known theorem, just like computations
performed using Turing Machine and~means that anything written in modern programming language is possible
to~translate into theoretical model of Minsky Machine (as well as into Turing Machine).

My goal is to build automatic translation from modern programming language (C++) to Minsky Machine.
}
% w pracach wielu autorow nazwiska mozna oddzielic poleceniem \and
\author         {Jadwiga Pokorska}
% w przypadku kilku promotorow, lub koniecznosci podania ich afiliacji, linie
% w ponizszym poleceniu mozna zlamac poleceniem \fmlinebreak
\advisor        {dr Jakub Michaliszyn}
%\date          {}                     % Data zlozenia pracy
% Dane do oswiadczenia o autorskim wykonaniu
\transcriptnum {247906}                     % Numer indeksu
\advisorgen    {dr Jakuba Michaliszyn} % Nazwisko promotora w dopelniaczu
%%%%%

%%%%% WLASNE DODATKOWE PAKIETY
%
%\usepackage{graphicx,listings,amsmath,amssymb,amsthm,amsfonts,tikz}
%
%%%%% WŁASNE DEFINICJE I POLECENIA
%
%\theoremstyle{definition} \newtheorem{definition}{Definition}[chapter]
%\theoremstyle{remark} \newtheorem{remark}[definition]{Observation}
%\theoremstyle{plain} \newtheorem{theorem}[definition]{Theorem}
%\theoremstyle{plain} \newtheorem{lemma}[definition]{Lemma}
%\renewcommand \qedsymbol {\ensuremath{\square}}
% ...
%%%%%

\begin{document}

%%%%% POCZĄTEK ZASADNICZEGO TEKSTU PRACY

\chapter{Introduction}

\ldots

\chapter {Specifications of languages}

\section {Brainfuck}
TODO.

\section {Turing Machine}

\subsection {Instructions}

Turing Machine consists of finite number of states, changes between them
and~potentially infinite tape, on~which it is able to~write and~read symbols.
It has no code sequence, program is a~set of state changes and~definition
of~initial state. There is special final state \texttt{END} that does not need
to~be defined to~be used.

There is predefined set of symbols used on the tape which is whole ASCII character set
extended with the same amount of additional (non-ASCII) symbols. Regular
ASCII characters have 7 bits (numbers 0-127), so tape symbols have 8 bits
allowing to hold regular ASCII and special characters (numbers 128-255).

There are few special definitions (so far using regular ASCII,
but it will be changed to use special characters instead):
\begin{itemize}
  \item \texttt{BLANK} which represents empty field on the tape
  \item \texttt{*} (ALL) defines all characters (both ASCII and special)
  \item \texttt{\#} (NOTHING) means no character which is used to say
                   that we do not want to write anything on tape during
                   this state change
  \item \texttt{\&} (NON-ZERO) defines all characters except 0
  \item \texttt{0} (ZERO) represents zero that is used for condotional jumps
                   (jump zero)
  \item \texttt{>} (NEXT-CHAR) allows to write on the tape next character
                   (increased by one), i.e. writes \texttt{g} if on the tape
                   was \texttt{f}
  \item \texttt{<} (PREV-CHAR) similarly as above but writes previous caracted,
                   i.e. writes \texttt{e} when seen \texttt{f}
\end{itemize}

Defining initial state:
\begin{verbatim}
START: <state name>
\end{verbatim}

Each state change looks as follows:
\begin{verbatim}
<state name> <symbol(s)> -> <target state> <head move> <symbol to write>
\end{verbatim}

\texttt{symbol(s)} is ASCII character including special definitions (currently
all special definitions use regular ASCII, so that it is easy to see
in standard text file), in future it will allow to get non-ASCII special
characters as well.

\texttt{head move} is one of: \texttt{L}, \texttt{R} or \texttt{-} which
steer the head to go left, right or stay in place, respectively.

\texttt{symbol to write} is any (currently ASCII) symbol. Currently there
is no mechanism to prevent writing any symbol from special definitions,
but~it will cause undefined behaviour of~the machine. The only symbol
from special definitions that is allowed to appear as \texttt{symbol to write}
is \texttt{\#} (NOTHING) meaning that symbol on the tape should not change.

\subsection {Extensions of theoretical model}

Standard input/output handling is done by allowing to read or write single
ASCII character from/to stdin/stdout. It is possible to~add additional
reading or~writing before moving head. It is done by modifying change
symbol \texttt{->} in state change definition.
\begin{itemize}
  \item Reading is done with \texttt{->*}, i.e. \texttt{state1 A ->* state2 R NOTHING}
        which means when seen symbol \texttt{A} in state \texttt{state1} we read
        from stdin one character, overwrite \texttt{A} to read value and move
        head one field right.
  \item Writing is done similarly with \texttt{->\^},
        i.e. \texttt{state1 A ->\^ state2 R NOTHING} which will print symbol
        \texttt{A} on stdout and move head right.
\end{itemize}

\subsection{Example}

Program that reads letter from stdin, writes this letter and next one into
stdout and second written letter was \texttt{B} or \texttt{b} then prints
\texttt{.} at the end as well, otherwise finishes.

\begin{verbatim}
START: s1
s1 ALL ->* s2 - NOTHING
s2 ALL ->^ s3 - NEXT_CHAR
s3 ALL -> s4 - NOTHING
s4 b ->^ s5 R NOTHING
s4 B ->^ s5 R NOTHING
s5 ALL -> s6 - .
s6 ALL ->^ s7 - NOTHING
s7 ALL -> END - NOTHING
\end{verbatim}

Note: If there is no state change defined for given configuration
(nothing matches) then it is assumed that machine gets to \texttt{END} state.
Because of this in the above example last instruction is not necessary.

\section {2 Stack Pushdown Automata}
Not yet defined.

\section {Minsky Machine}

\subsection {Instructions}
Minsky Machine (currently) consists of finite number of counters
each holding one non-negative integer and sequence of instructions.
Each instruction might be prefixed by a label which will be used for
code execution and allow to jump into desired code parts. (The idea is similar
to RAM machine.)

Each of the instructions is one of the following ones:
\begin{itemize}
  \item \texttt{INC <number>} increases given counter
  \item \texttt{DEC <number>} decreases given counter
  \item \texttt{PRINT <number>} prints state of given counter to standard output
  \item \texttt{READ <number>} reads number from standard input into given counter
  \item \texttt{JZ <number> <label>} if value of given counter is 0 then execution
        will continue from place in code with given label
\end{itemize}

\subsection {Example}
Program that reads number $n$ from stdin and writes $2 \cdot n$ to stdout.

\begin{verbatim}
      READ 1
loop: JZ 1 end
      DEC 1
      INC 2
      INC 2
      JZ 3 loop
end:  PRINT 2
\end{verbatim}

%%%%% BIBLIOGRAFIA

%\begin{thebibliography}{1}
%\bibitem{example} \ldots
%\end{thebibliography}

\end{document}
